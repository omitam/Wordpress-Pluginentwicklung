\section{Zusammenfassung und Ausblick}\label{fazit}
Abschließend lässt sich formulieren, dass diese Dokumentation als Einleitung zur Entwicklung für Wordpress-Plugins darstellt. \newline
Allerdings sind weitere Literatur und Internetquellen von notwendig, um in bestimmten Themengebieten tiefer einzusteigen.\newline
Dafür sind im Quellverzeichnis entsprechende Literatur und Internetquellen angegeben, welche aus der Sicht der Autoren sind als zweckmäßig erwiesen hat und hier entsprechend verwendet wurde. Trotz der Themenbreite wurde versucht, jedes Thema zu erläutern. Dies sollte gewährleisten, dass Grundkenntnisse für die Entwicklung vorhanden sind.\newline
Das Ziel des Tutorials war es, einen Überblick über einzelne Themengebiete der Wordpress-Plugin-Entwicklung aufzuzeigen. Dabei wurde sich an einen Ablauf der Entwicklung gehalten, welcher auch in der Praxis weit verbreitet ist. Genauer gesagt, wurde von der Einleitung, allgemeinen Grundlagen zur Entwicklung von Wordpress-Plugins über die Menüerstellung und Shortcodes schlussendlich zu Datenbankzugriffen, Formularen, der Internationalisierung und Lokalisierung bis zu Installations- und Deinstallationsroutinen inhaltlich beschrieben, wie Plugins entwickelt werden. Es sollte weiterhin darauf geachtet werden, dass zwar ab dem Kapitel 4 (Menüerstellung) die Grundlagen abgeschlossen ist und theoretisch in beliebiger Reihenfolge weiter verfolgt werden kann.  Dies ist zwar möglich, ist aber aus Sicht der Autoren nicht zu empfehlen.\newline
Aus diesem Dokument lassen sich verschiedene weitere Dokumente erzeugen lassen, da es sich um ein Grundlagentutorial handelt. Dabei können sich auf einzelne Kapitel bezogen werden und eventuell zu bestimmten Problematiken Lösungsvorschläge- oder alternativen formuliert werden.\newline
Insgesamt lässt sich formulieren, dass zwar die Wordpressentwicklung nicht so komplex ist, wie es auf den ersten Blick scheint. Trotzdem sind fundamentale Programmierkenntnisse notwendig - gerade was dir Programmierung in HTML und PHP angeht. Andernfalls entstehen Problematiken, die nicht direkt mit der Wordpress-Entwicklung zu tun haben und entsprechend mehr Zeit in der Entwicklung kosten.\newline
Bei der aktiven Entwicklung ist es wichtig darauf zu achten, dass Wordpress regelmäßig neue Versionen ihrer Software veröffentlicht. So ist für eine Pluginentwicklung nicht nur entscheidend, für welche aktuelle Version das Plugin geschrieben wird, sondern auch, ob das Plugin für nächste Versionen bereit sein soll. Dazu sollte auf möglicherweise neue Funktionen oder auf jene, welche nicht benutzt werden, geachtet werden. Das würde auch mögliche Probleme bei der Weiterentwicklung einschränken.\newline
Auch lässt sich feststellen, dass es gewisse Normen und Best-Practice-Ansätze in der Pluginentwicklung von den verschiedenen Autoren gibt. Diese wurden in diesem Umfeld nicht komplett 1:1 umgesetzt. Dies ist damit zu erklären, dass dies das erste Plugin für Wordpress aus Sicht der Autoren war und in Zukunft anders gehandhabt wird.\newline
Gerade bei größerer Softwareentwicklungsprozessen sind auf gewisse Standards zu achten - gerade auch in dem Hinblick dieses aus \url{wordpress.com} hochzuladen und zu veröffentlichen. \newline
Dies wäre beuispielsweise ein Kapitel, welches hier nicht angesprochen wurde: Die Veröffentlichung über \url{wordpress.org/plugins} lässt sich daher auf der offiziellen Pluginseite und den Literaturangaben zum Selbststudium nachlesen und das Plugin entsprechend anpassen. 
Gerade in den angegebenen Büchern finden sich schon zu Anfang entsprechende Kapitel und Abschnitte, welche zusätzlich zu dem hier beschriebenen Einstieg hervorragend geeignet sind. Auch für speziellere Themen sind diese zu empfehlen.\newline
Während der Entwicklung und Verfassung des Tutorials, hatten die Autoren zwar viel Spaß bedingt durch dieses interessante Thema, es gab allerdings auch Hindernisse. Diese wurden durch intensive Recherche und Ausprobieren gelöst. Beispielsweise durch Aufsetzung einer zweiten Wordpress-Test-Umgebung, um die Hauptentwicklung nicht zu gefährden.\newline
Dabei ist ein wichtiger Punkt, dass eine Versionskontrollsoftware eingesetzt wird, um die Konsistenz des Quellcodes zu gewährleisten. Dies ist auch von Vorteil, um eine Übersicht über verschiedene Projektphasen zu bekommen.\newline
Die Autoren wünschen eine interessante Pluginentwicklung und sind für Kritik und Verbesserungsvorschläge immer offen.\newline\ \newline
\emph{Timo Amling (Autor): timo.amling\texttt{@}fh-koeln.de, \newline Anatol Tissen (Entwicklung): anatol.tissen\texttt{@}fh-koeln.de und \newline Ludger Schönfeld (Entwicklung und Co-Autor): ludger.schoenfeld\texttt{@}fh-koeln.de}
